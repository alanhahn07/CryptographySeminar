\documentclass{beamer}

\usepackage{amsmath}
\usepackage{amsthm}
\usepackage{amsfonts}
\usepackage{amssymb,enumerate}
\usepackage{amsthm,stmaryrd}
\usepackage[all]{xy}
\usepackage{hyperref}
\usepackage{xcolor}
\usepackage{tikz}
\usepackage{scrextend}
\usepackage{apacite}
\usetikzlibrary{shapes.geometric}


%
% Choose how your presentation looks.
%
% For more themes, color themes and font themes, see:
% http://deic.uab.es/~iblanes/beamer_gallery/index_by_theme.html
%
\mode<presentation>
{
  \usetheme{Madrid}      % or try Darmstadt, Madrid, Warsaw, ...
  \usecolortheme{default} % or try albatross, beaver, crane, ...
  \usefonttheme{default}  % or try serif, structurebold, ...
  \setbeamertemplate{navigation symbols}{}
  \setbeamertemplate{caption}[numbered]
  %\setbeamertemplate{theorems}[numbered]
} 

\usepackage[english]{babel}
\usepackage[utf8x]{inputenc}
%\usepackage[utf8]{inputenc}
%\usepackage{enumitem}
\usepackage{amsmath}
\usepackage{amsfonts}
\usepackage{amssymb,enumerate}
\usepackage{amsthm,stmaryrd}
\usepackage{float}
\usepackage{graphicx}
\usepackage{verbatim}
\usepackage{subcaption}
\usepackage{mathtools}
\usepackage{fancyvrb}
\newtheorem{prop}{Proposition}
\newtheorem{prop1}{Proposition 1}
\newtheorem{prop2}{Proposition 2}
\newtheorem{defn}[prop]{Definition}
\newtheorem{lem}{Lemma}
\newtheorem{ex}{Example}
\newtheorem{n}{Note}
\newtheorem{cor}{Corollary}
\newtheorem{BA}{Buchberger's Algorithm}
\newtheorem{gbc1}{GB Criterion 1}
\newtheorem{gbc2}{GB Criterion 2}
\newtheorem{gbc3}{GB Criterion 3}
\newtheorem{defsnota}{Definitions and Notation}
\newtheorem{thm}{Theorem}
\newtheorem{fainf}{Facts about Ideals in Number Fields}
\newtheorem{rmk}{Remark}
\newtheorem{aoam}{Analysis of Add and Multiply}
\newtheorem{PN}{Projective Nullstellensatz}
\newtheorem{AN}{Recall Affine Nullstellensatz}
\newtheorem{IncStep}{Inclusion Step}
\newtheorem{PC}{The Partition Class}
\newtheorem{SPIP}{Small Principal Ideal Problem}
\newtheorem{cscheme}{Somewhat Homomorphic Scheme}
\usepackage{mathrsfs}


\title[FHE Implementation]{An implementation of FHE with small ciphertext and key size through a modification of Gentry}
\author{Alan R. Hahn}
\institute{Technische Universit{\"a}t Kaiserslautern}
\date{Jan 2020}

\begin{document}

\begin{frame}
  \titlepage
\end{frame}


\begin{frame}
\begin{small}
\begin{SPIP}[SPIP]
Given a principal ideal $\mathfrak{a}$ in two element representation, compute a "small" generator of the ideal.
\end{SPIP}
\pause
\begin{itemize}
\item $N^{O(N)}\cdot \sqrt{\text{min}(A,R)}\cdot |\Delta|^{O(1)}$
\item exp$(O(N \text{ log } N)\cdot \sqrt{\text{log}(\Delta)\cdot \text{loglog} (\Delta)})$
\end{itemize}
\end{small}
\end{frame}




\begin{frame}
\begin{small}
%\frametitle{Notation and Set Up}
\begin{defsnota}
%\begin{itemize}
For $g(x) = \sum_{i = 0}^{t}g_ix^i\in\mathbb{Q}[x]$, define
%\vspace{3mm}
\\$$||g(x)||_2 = \sqrt{\sum_{i = 0}^{t}g_i^2} \text{ \hspace{2mm}    and \hspace{2mm}    } ||g(x)||_{\infty} = \max\limits_{i = 0,...,t}|g_i|.$$
\pause
For $r>0$, define
$$B_{2,N}(r) = \left\{ \sum_{i = 0}^{N-1}a_ix^i: \sum_{i = 0}^{N-1}a_i^2\leq r^2\right\},$$
$$B_{\infty, N}(r) = \left\{ \sum_{i = 0}^{N-1}a_ix^i: -r\leq a_i\leq r\right\},$$
$$B_{\infty, N}^+(r) = \left\{ \sum_{i = 0}^{N-1}a_ix^i: 0\leq a_i\leq r\right\}.$$
\vspace{2mm}
\pause
\\Note $B_{2,N}(r) \subset B_{\infty, N}(r) \subset B_{2,N}(\sqrt{N}\cdot r)$.

%\item[Note] Hello There!
%\end{itemize}
\end{defsnota}
\end{small}
\end{frame}

\begin{frame}
\begin{small}
\begin{defsnota}[Continued]
Denote by $a\leftarrow b$ the assignment of the value of b to the value of a. 
\\Denote by $a\leftarrow_RA$, for a set $A$, the selection of $a$ from $A$ using a uniform distribution. 
\\For $m\in \mathbb{Z}_{Odd}$, reductions modulo $m$ result in a value in the range $[-(m-1)/2,(m-1)/2].$
\end{defsnota}
\pause
\begin{fainf}
Let $K = \mathbb{Q}(\theta)$, with $F(\theta) = 0$ for some monic irreducible $F\in \mathbb{Z}[x]$ of degree $N$.
\\Consider $\mathbb{Z}[\theta] \subset \mathcal{O}_K$; the scheme works with ideals of $\mathbb{Z}[\theta]$ coprime to  $[\mathcal{O}_K : \mathbb{Z}[\theta]].$ Such ideals can be generated by two elements.
%\pause
%\\Indeed, for a rational prime p,
%$$F(x) = \prod_{i = 1}^t F_i(x)^{e_i} \hspace{2mm} \text{(mod p)},$$
%so that for ideals lying above a rational prime $p$,  $p$ not dividing $[\mathcal{O}_K : \mathbb{Z}[\theta]]$, the %prime ideals dividing $p\mathbb{Z}[\theta]$ are given by 
%$$\mathfrak{p}_i = \langle p,F_i(\theta)\rangle.$$
\end{fainf}
\end{small}
\end{frame}

\begin{frame}
\begin{small}
\begin{fainf}[Continued]
Let $K = \mathbb{Q}(\theta)$, with $F(\theta) = 0$ for some monic irreducible $F\in \mathbb{Z}[x]$ of degree $N$.
\\Consider $\mathbb{Z}[\theta] \subset \mathcal{O}_K$; the scheme works with ideals of $\mathbb{Z}[\theta]$ coprime to  $[\mathcal{O}_K : \mathbb{Z}[\theta]].$ Such ideals can be generated by two elements.
\\Indeed, for a rational prime p,
$$F(x) = \prod_{i = 1}^t F_i(x)^{e_i} \hspace{2mm} \text{(mod p)},$$
so that for ideals lying above a rational prime $p$,  $p$ not dividing $[\mathcal{O}_K : \mathbb{Z}[\theta]]$, the prime ideals dividing $p\mathbb{Z}[\theta]$ are given by 
$$\mathfrak{p}_i = \langle p,F_i(\theta)\rangle.$$ For $F_i(x)$ of degree 1, reduction modulo $\mathfrak{p}_i$ produces a homomorphism 
$$\iota_{\mathfrak{p}_i }:\mathbb{Z}[\theta]\rightarrow \mathbb{F}_p,$$
and $\mathfrak{p}_i = \langle p, \theta - \alpha \rangle$, where $\alpha$ is a root of $F(x)$ modulo $p$.
\\Given $\chi = \sum_{i = 0}^{N-1}c_i\theta^i$, $\iota_{\mathfrak{p}_i }$ then corresponds to evaluation of $\chi(\theta)$ in $\alpha$ modulo $p$.
\end{fainf}
\end{small}
\end{frame}

\begin{frame}
\begin{scriptsize}
%\frametitle{Somewhat Homomorphic Scheme}
%\begin{cscheme}
\begin{center}
$\mathbf{Somewhat\text{ }Homomorphic\text{ }Scheme}:$ Parameters $N, \eta, \mu$
\end{center}
\begin{itemize}
\item $\bold{KeyGen()}:$
\\- Set plaintext space $\mathcal{P} = \{0,1\}$
\\- Choose monic, irreducible $F(x)\in \mathbb{Z}[x]$ of degree $N$
\\- Repeat until $p$ prime: 
\\\hspace{3mm} - $S(x) \leftarrow_R B_{\infty, N}(\eta/2)$
\\\hspace{3mm} - $G(x) \leftarrow 1+2\cdot S(x)$
\\\hspace{3mm} - $p \leftarrow resultant(G(x), F(x))$
\\- $D(x) \leftarrow gcd(G(x),F(x))$ over $\mathbb{F}_p[x]$
\\- Denote by $\alpha \in \mathbb{F}_p$ the unique root of $D(x)$
\\- Apply XGCD-algorithm over $\mathbb{Q}[x]$ to obtain $Z(x) = \sum_{i = 0}^{N-1}z_ix^i\in \mathbb{Z}[x]$ such that 
\\\hspace{2mm}$Z(x)\cdot G(x) = p  \hspace{2mm} \text{(mod F(x))}$
\\- $B\leftarrow z_0 \hspace{2mm} \text{(mod 2p)}$
\\\vspace{0.5 mm} The public key $PK = (p,\alpha)$, the private key $SK = (p,B)$
\end{itemize}

%------------------------------------------------------------------------------
\begin{minipage}{.45\textwidth}
\begin{itemize}
\item $\mathbf{Encrypt}(M, PK):$
\\- Parse $PK$ as $(p,\alpha)$
\\- If $M\notin \{0,1\}$, abort
\\- $R(x)\leftarrow_R B_{\infty, N}(\mu/2)$
\\- $C(x) \leftarrow M + 2\cdot R(x)$
\\- $c\leftarrow C(\alpha) \hspace{2mm} \text{(mod p)}$
\\- Output $c$

\item $\mathbf{Add}(c_1,c_2, PK):$
\\- Parse $PK$ as $(p,\alpha)$
\\- $c_3\leftarrow (c_1+c_2) \hspace{2mm} \text{(mod p)}$
\\- Output $c_3$
\end{itemize}
\end{minipage}
%\scshape
\begin{minipage}{.45\textwidth}
\begin{itemize}
\item $\mathbf{Decrypt}(c, SK):$
\\- Parse $SK$ as $(p,B)$
\\- $M\leftarrow (c - \lfloor c\cdot B/p\rceil) \hspace{2mm} \text{(mod 2)}$
\\- Output $M$
\\\vspace{9 mm}
\item $\mathbf{Mult}(c_1,c_2, PK):$
\\- Parse $PK$ as $(p,\alpha)$
\\- $c_3\leftarrow (c_1\cdot c_2) \hspace{2mm} \text{(mod p)}$
\\- Output $c_3$

\end{itemize}
\end{minipage}
%\end{cscheme}
\end{scriptsize}
\end{frame}

\begin{frame}
\begin{footnotesize}

\begin{aoam}

Recall decryption of $c = C(\alpha)$ requires $C(x) = M + 2\cdot R(x)\in B_{\infty, N}(r_{\text{Dec}}).$
\pause
\\Let  $c_1$, $c_2$ and $C_1(x) = M_1+N_1(x), C_2(x) = M_2+N_2(x)$ denote two ciphertexts and their corresponding polynomials, with $C_i(x)\in B_{\infty, N}(r_i)$. 
\pause
\\Then for the addition and multiplication of $C_i(x)$
\begin{gather*}
C_3(x) = M_3+N_3(x) = (M_1+N_1(x)) + (M_2+N_2(x)),
\\C_4(x) = M_4+N_4(x) = (M_1+N_1(x)) \cdot (M_2+N_2(x)),
\end{gather*}
$C_3(x)\in B_{\infty, N}(r_1+r_2)$, $C_4(x)\in B_{\infty, N}(\delta_{\infty}\cdot r_1\cdot r_2)$ \hspace{2mm}$(||g(x)\cdot h(x)||_{\infty} \leq \delta_{\infty}\cdot||g(x)||_{\infty}\cdot||h(x)||_{\infty})$.
\pause
\\Thus after executing a circuit of multiplicative depth d for an initial $C(x)\in B_{\infty, N}(\mu)$, we get the corresponding polynomial $C'(x)\in B_{\infty, N}(r)$ with $$r \approx (\delta_{\infty}\cdot \mu)^{2^{d}}.$$
\\Then $r \approx (\delta_{\infty}\cdot \mu)^{2^{d}} \leq r_{Dec} \Rightarrow$
$$d\text{log }2 \leq \text{log log } r_{Dec} - \text{log log}(\delta_{\infty}\cdot \mu).$$
\end{aoam}
\end{footnotesize}
\end{frame}


%\begin{frame}
%\end{frame}






\end{document}